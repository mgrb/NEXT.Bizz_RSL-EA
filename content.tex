\begin{document}
% -----------------------------------------------
% INTRODUCTION
% -----------------------------------------------
\section{INTRODUÇÃO}

O mercado está cada vez mais exigente em relação à qualidade dos profissionais de tecnologia da informação e comunicação (TIC). A indústria reclama que a universidade atual não consegue oferecer currículos que resolvam as questões práticas e reais do desenvolvimento de software [1-2]. Com o crescimento explosivo da internet e a penetração de software em quase todos os aspectos da nossa vida, a necessidade de desenvolvedores de software qualificados para construir software de qualidade é evidente. O mercado reivindica instituições de ensino para formar futuros empregados de tecnologia com qualidade. No entanto, as universidades enfatizam a educação em longo prazo, em vez de treinamento que resultem em habilidades de curto prazo.

Diante da necessidade discutida acima, surge o conceito de fábrica de software no processo de aprendizagem e qualificação dos estudantes. A ideia é que a fábrica de Software seja combinada aos cursos tradicionais de TI para cobrir os temas da área. Tvedt, Tesoriero e Gary [3] propõem aplicar o conceito de fábrica de software para desenvolver um projeto de grande escala com estudantes de diferentes cursos. Os alunos trabalham juntos com responsabilidades diferentes. O problema deste formato é a dificuldade de integrar os diferentes cursos e seus alunos, professores e datas - o que torna difícil implementar um conjunto de cursos com este formato. Apenas uma parte da ideia de fábrica de software é aplicada, pois não existe uma estrutura bem definida do processo ou uma avaliação quantitativa.

Logo, a principal motivação para aplicar o conceito de fábrica de software em cursos de nível superior de TI, é criar um ambiente que esteja mais perto de um verdadeiro cenário de desenvolvimento de software. As características de uma fábrica de software apresentam algumas outras vantagens, por exemplo: os estudantes usam seus conhecimentos em um ambiente controlado, que permite um acompanhamento mais de perto pelos professores; e processos bem definidos, o que facilita a compreensão do que deve ser feito. Assim, a finalidade da fábrica de software é fornecer aos estudantes uma experiência prática em desenvolvimento de software. Os alunos devem ganhar experiência no negócio, bem como a experiência técnica. Apesar dos potenciais benefícios e a crescente prática de fábrica de software em cursos de nível superior de TI, não se tem um acervo científico acumulado do conhecimento sobre os benefícios e as limitações de quais metodologias são praticadas para conduzir uma fábrica de software em cursos de nível superior de TI.

Por esta razão, este trabalho está interessado em compreender as metodologias existentes e como essa prática contribui para a condução da aplicação de fábrica de software, a partir da análise de estudos anteriores na engenharia de software na educação. Para alcançar este objetivo, foi realizada uma revisão sistemática da literatura, a fim de identificar, analisar e sintetizar as evidências produzidas sobre este assunto ao longo dos anos, tentando responder as questões de pesquisa definidas na próxima seção.

Desta forma, este artigo apresenta uma Revisão Sistemática da Literatura (RSL) que teve como intuito descobrir quais objetivos, necessidades de informação e abordagens são utilizados no contexto de fábrica de software aplicados em ambientes educacionais. As estratégias, metodologias e resultados preliminares da RSL também serão apresentados com base nos seguintes tópicos:

\begin{itemize}
    \item Engenharia de Software Baseada em Evidência;
    \item Metodologia de uma RSL: Planejamento, Condução e Apresentação;
    \item Resultados Preliminares: Planejamento e Condução.
\end{itemize}

Vale ressaltar que os estudos identificados no contexto desta RSL serão utilizados para fundamentar a proposta quanto às metodologias que podem ser adotadas para condução real de uma fábrica de software em cursos de nível superior de TI.

Além desta seção introdutória, este trabalho está estruturado da seguinte forma: a Seção 2 apresenta a fundamentação teórica sobre engenharia de software baseada em evidências; a Seção 3 apresenta detalhes da metodologia da Revisão Sistemática da Literatura usada neste trabalho; a Seção 4 apresenta a Revisão Sistemática realizada neste trabalho, tratando sobre o planejamento e condução; na Seção 5 são apresentados os resultados obtidos com os trabalhos selecionados na revisão sistemática; e a Seção 6 apresenta algumas conclusões sobre o trabalho realizado.



% -----------------------------------------------
% BACKGROUND AND RELATED WORK
% -----------------------------------------------
\section{ENTERPRISE ARCHITECTURE COMO RESPOSTA AOS DESAFIOS ESTRATÉGICOS DAS ORGANIZAÇÕES}
\label{sec:trabalhosRelacionados}

Segundo Mafra e Travassos [4], o atraso no estado da arte da indústria de software é evidenciado pelo fato de que frequentemente a indústria de software é acometida por diversos anúncios de “cura” para os mais variados problemas, o que evidencia que em 2006 a situação não havia mudado consideravelmente em relação à década anterior. Com isto, surgem dúvidas tais como de qual solução deve-se escolher, onde buscar fundamentação para auxiliar essa escolha ou como saber a precedência dessa solução. Ainda segundo os mesmos autores, os estudos experimentais consistem em uma importante maneira de se obter maiores informações a respeito de novas tecnologias, metodologias e boas práticas quanto ao desenvolvimento de software. 
Segundo Kitchenham [5], uma Revisão Sistemática da Literatura é uma forma de avaliar e interpretar todas as pesquisas disponíveis, referentes a uma questão de investigação particular, área temática ou fenômeno de interesse. Assim, pode-se definir esta revisão como uma revisão abrangente e não tendenciosa. Kitchenham afirma, ainda, que as revisões sistemáticas têm por objetivo apresentar uma avaliação justa de um tópico de investigação, usando uma metodologia confiável, rigorosa e auditável. 
Em vista disso, Mafra e Travassos [4] afirmam que para atingir um nível adequado de evidência a respeito da caracterização de uma determinada tecnologia em uso, a engenharia de software baseada em evidência deve fazer uso basicamente de dois tipos de estudos [4]:

\begin{itemize}
    \item Estudos primários, que são os estudos que visam caracterizar uma determinada tecnologia em uso dentro de um contexto específico, onde se encontram os surveys e os estudos de caso;
    \item Estudos secundários, que são os estudos que visam identificar, avaliar e interpretar todos os resultados relevantes a um determinado tópico de pesquisa, fenômeno de interesse ou questão de pesquisa, onde se encontram as revisões sistemáticas.
\end{itemize}

Um dos principais métodos da engenharia de software baseada em evidências são as RSL, classificadas como estudos secundários, já que dependem dos estudos primários utilizados para revelar evidências e construir conhecimento [6-7].


% -----------------------------------------------
% METHOD
% -----------------------------------------------
\section{REVISÃO SISTEMÁTICA DA LITERATURA}

Nesta seção será apresentado um protocolo de busca em fontes da academia com o intuito de observar e identificar o estado de arte atual, vantagens, desvantagens e frameworks adotados na implementação e uso da Enterprise Architecture (EA). Como ponto de partida, foi estabelecida uma metodologia dividida em 5 etapas, conforme a Fig. 1.

% -- INSERT IMG: Figura 1. Processo cíclico para refinamento da busca

Uma revisão sistemática deve ser conduzida por meio de um protocolo pré-estabelecido para garantir que essa revisão tenha de fato valor científico, assim como a possibilidade de repetição, pois caso isso não possa reproduzido, as revisões tornam-se informais pois criam uma dependencia dos revisores que a conduziram, destaforma, diminuem seu grau de confiabilidade.
Para Kitchenham [5], antes da realização de uma revisão sistemática, os pesquisadores devem garantir sua necessidade. Da mesma forma que o protocolo utilizado, seja capaz de responder a algumas questões:

\begin{itemize}
    \item Quais são os objetivos da revisão?
    \item Quais fontes foram pesquisadas para identificar os estudos primários? 
    \item Houveram restrições?
    \item Foram aplicados critérios de inclusão e exclusão?
    \item Quais critérios foram utilizados para avaliação dos estudos primários?
    \item Qual a forma de extração dos dados obtidos durante os estudos primários?
    \item De que forma os dados foram sintetizados?
    \item Quais foram as diferenças entre os estudos pesquisados?
    \item De que forma os dados foram combinados?
\end{itemize}

Segundo Travassos e Biolchini [8], por meio de alguns passos, confirma e acrescenta novas informações sobre o que a fase de planejamento deve contemplar:

\begin{itemize}
    \item Objetivos da pesquisa devem ser listados;
    \item Questões de pesquisa formuladas (strings de busca estruturada);
    \item Definição dos métodos que serão utilizados para executar a revisão, assim como o processo para analisar os dados obtidos;
    \item Fazer o planejamento para seleção dos estudos a partir das fontes definidas;
    \item Proceder com a definição do protocolo da revisão o qual será documentado e disponibilizado.
\end{itemize}

\subsection{Perguntas de Pesquisa}
Nós usamos a seguinte questão para guiar nossos processos de pesquisa e seleção:

\begin{enumerate}[start=0,label={R\arabic*):}]
    \item \label{pp:perguntaPrincipal} Qual o estado da arte da arquitetura corporativa? 
\end{enumerate}

Nós também usamos xx perguntas especificas para guiar e estruturar a extração, analise e sintese das evidencias:

\begin{enumerate}[label={R\arabic*):}]
    \item Em quais contextos estão sendo aplicados a Enterprise Architecture?
    \item Quais as vantagens e beneficios identificados nos estudos da Enterprise Architecture pesquisados?
    \item Quais as desvantagens e desafios identificados nos estudos da Enterprise Architecture pesquisados?
    \item Quais as definições encontradas para a Enterprise Architecture nos estudos pesquisados?
    \item Quais frameworks de Enterprise Architecture estão sendo utilizados nos estudos pesquisados?
\end{enumerate}

\subsection{Base de Dados e Estratégia de Busca}

Foi realizada uma busca automática, em seis mecanismos de busca e indexadores (Tabela~\ref{tab:engenhosBusca}), utilizando uma string de busca baseada nos termos gerais extraídos do \ref{pp:perguntaPrincipal} e em sinônimos de Enterprise Architecture encontrados na literatura discutida na Seção~\ref{sec:trabalhosRelacionados}. O processo de pesquisa automatizado realizado até outubro de 2018 recuperou mais de 1545 documentos.

\begin{table}[H] 
    \label{tab:engenhosBusca}
    \centering
    \caption{Engenhos de Busca Automáticos}
    \begin{tabular}{@{}ll@{}}
        \toprule
        Engenho de Busca    & Link                                  \\ \midrule
        ACM Digital Library & http://dl.acm.org/                    \\
        IEEEXplore          & http://www.ieeexplore.ieee.org/Xplore \\
        Scopus              & http://www.scopus.com/home.url        \\
        Science Direct      & http://www.sciencedirect.com/         \\
        Springer            & http://www.springer.com.br/           \\
        Emerald             & https://www.emeraldinsight.com/       \\ \bottomrule
    \end{tabular}
\end{table}

especificar a string de busca

\subsection{Critérios de Inclusão e Exclusão}

Nosso conjunto inicial de estudos encontrados partil de XXX estudos. 

A partir do conjunto inicial de xxx artigos, foram selecionados estudos apresentando conceitos, teorias, diretrizes, discussões, lições aprendidas e relatos de experiência sobre as práticas da Arquitetura Corporativa na industria (critérios de inclusão). Excluímos artigos que foram enquadrados em qualquer um dos sete critérios de exclusão:

\begin{enumerate}
    \item Escrito em qualquer idioma, exceto inglês;
    \item Não acessível na Web;
    \item Trabalhos convidados, Palestras, relatórios de oficinas, livros, mapeamentos sistemáticos, revisões sistemáticas, teses e dissertações;
    \item Documentos incompletos, rascunhos, slides de apresentações e resumos estendidos; 
    \item Abordar outras áreas além da ciência da computação (por exemplo, negócios e administração, ciências sociais, saúde e outros);
    \item Estudos citando ou apenas referenciando trabalhos sobre arquitetura corporativa, mas não abordando a arquitetura corporativa e evidências;
    \item Artigos que não apresentam um estudo de caso sobre a prática de arquitetura corporativa.
\end{enumerate}

\subsection{Seleção de Estudos}
\subsection{Extração de Dados}
\subsection{Síntese dos Dados}

% -----------------------------------------------
% RESULTS
% -----------------------------------------------
\section{RESULTADOS}


% -----------------------------------------------
% DISCUSSION
% -----------------------------------------------
\section{DISCUSSÃO}

% -----------------------------------------------
% CONCLUSIONS
% -----------------------------------------------
\section{CONCLUSÃO}
\subsection{Limitações da Revisão}

\end{document}